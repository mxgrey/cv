%%%%%%%%%%%%%%%%%
% This is an sample CV template created using altacv.cls
% (v1.1.5, 1 December 2018) written by LianTze Lim (liantze@gmail.com). Now compiles with pdfLaTeX, XeLaTeX and LuaLaTeX.
%
%% It may be distributed and/or modified under the
%% conditions of the LaTeX Project Public License, either version 1.3
%% of this license or (at your option) any later version.
%% The latest version of this license is in
%%    http://www.latex-project.org/lppl.txt
%% and version 1.3 or later is part of all distributions of LaTeX
%% version 2003/12/01 or later.
%%%%%%%%%%%%%%%%

%% If you need to pass whatever options to xcolor
\PassOptionsToPackage{dvipsnames}{xcolor}

%% If you are using \orcid or academicons
%% icons, make sure you have the academicons
%% option here, and compile with XeLaTeX
%% or LuaLaTeX.
% \documentclass[10pt,a4paper,academicons]{altacv}

%% Use the "normalphoto" option if you want a normal photo instead of cropped to a circle
% \documentclass[10pt,a4paper,normalphoto]{altacv}

\documentclass[10pt,a4paper,ragged2e]{altacv}

\usepackage{hyperref}

%% AltaCV uses the fontawesome and academicon fonts
%% and packages.
%% See texdoc.net/pkg/fontawecome and http://texdoc.net/pkg/academicons for full list of symbols. You MUST compile with XeLaTeX or LuaLaTeX if you want to use academicons.

% Change the page layout if you need to
\geometry{left=1cm,right=9cm,marginparwidth=6.8cm,marginparsep=1.2cm,top=1.25cm,bottom=1.25cm}

% Change the font if you want to, depending on whether
% you're using pdflatex or xelatex/lualatex
\ifxetexorluatex
  % If using xelatex or lualatex:
  \setmainfont{Lato}
\else
  % If using pdflatex:
  \usepackage[utf8]{inputenc}
  \usepackage[T1]{fontenc}
  \usepackage[default]{lato}
\fi

% Change the colours if you want to
%\definecolor{HeadingColor}{HTML}{263f44}
%\definecolor{AccentColor}{HTML}{015668}

\definecolor{HeadingColor}{HTML}{3e4a61}
\definecolor{AccentColor}{HTML}{c24d2c}

\definecolor{SlateGrey}{HTML}{2E2E2E}
\definecolor{LightGrey}{HTML}{666666}
\colorlet{heading}{HeadingColor}
\colorlet{accent}{AccentColor}
\colorlet{emphasis}{SlateGrey}
\colorlet{body}{LightGrey}

% Change the bullets for itemize and rating marker
% for \cvskill if you want to
\renewcommand{\itemmarker}{{\small\textbullet}}
\renewcommand{\ratingmarker}{\faCircle}

%% cv.bib contains your publications
\addbibresource{cv.bib}

\begin{document}
\name{Grey}
\tagline{Senior Software Engineer}
\personalinfo{%
  % Not all of these are required!
  % You can add your own with \printinfo{symbol}{detail}
  \location{Singapore}
  \homepage{\href{https://greyxmike.info/}{https://greyxmike.info/}}
  \github{\href{https://github.com/mxgrey}{@mxgrey}}
  \email{greyxmike@gmail.com}
  \phone{+65 8877-6696}
  \phone{+1 (847) 530-1093}
  %% You MUST add the academicons option to \documentclass, then compile with LuaLaTeX or XeLaTeX, if you want to use \orcid or other academicons commands.
  % \orcid{orcid.org/0000-0000-0000-0000}
}

%% Make the header extend all the way to the right, if you want.
\begin{fullwidth}
\makecvheader
\end{fullwidth}

%% Depending on your tastes, you may want to make fonts of itemize environments slightly smaller
% \AtBeginEnvironment{itemize}{\small}

%% Provide the file name containing the sidebar contents as an optional parameter to \cvsection.
%% You can always just use \marginpar{...} if you do
%% not need to align the top of the contents to any
%% \cvsection title in the "main" bar.
\cvsection[sidebar-p1]{Employment \& Research}

\cvevent{Robotics Middleware Framework for Healthcare}{Open Robotics}{May 2018 -- Ongoing}{Singapore}
Designing and implementing open source software and open specifications to enable and enhance integration of heterogeneous robotics systems, with a focus on the healthcare sector. Healthcare facilities have remarkably complex logistical needs, which could be solved using sufficiently well-integrated robotics systems. However, no single vendor can currently provide a complete solution for what is needed, so as a vendor-neutral third-party open source middleware developer, we are uniquely positioned to define an open specification for how various vendors can integrate their systems into an effective coherent solution. 

\divider

\cvevent{Robotics Software and Physics Simulation}{Open Robotics}{July 2017 -- May 2018}{Mountain View, CA, USA}
Designed and implemented open source software to facilitate robotics research, development, testing, and deployment. This includes middleware, simulation tools, autonomous fleet management libraries, and CLI/GUI utilities.

\divider

\cvevent{Humanoid Robot Motion Planning}{Graphics Lab \& AMBER Lab}{August 2015 -- July 2017}{Atlanta, GA, USA}
Developed algorithms to efficiently perform locomotion and manipulation planning for humanoid robotic platforms.

\divider

\cvevent{Humanoid Robot Teleoperation}{Humanoid Robotics Lab}{August 2011 -- August 2015}{Atlanta, GA, USA}
Developed software systems to allow human users to teleoperate humanoid robots to perform complex tasks which reflect the needs of various disaster scenarios.

\divider

\cvevent{DARPA Robotics Challenge Trials}{Team DRC-Hubo}{August 2012 -- December 2014}{Atlanta, GA, USA}
Worked on a multi-institute team to participate in a DARPA-sponsored competition. Helped to develop and maintain core software systems at all levels of the teleoperation pipeline, and co-piloted several of the DRC tasks.

\divider

\cvevent{Swarm Control Testbed}{Aerospace Robotics and Control Lab}{May 2010 -- August 2011}{Urbana-Champaign, IL, USA}
Designed and implemented a closed-loop hardware testbed for examining control algorithms on autonomous flying swarms and small-scale model aircraft.

\clearpage

\cvsection[sidebar-p2]{Publications}

\nocite{*}

%% Currently I do not have any books, so we'll leave out this sub-section until I do
%\printbibliography[heading=pubtype,title={\printinfo{\faBook}{Books}},type=book]
%\divider

\printbibliography[heading=pubtype,title={\printinfo{\faGraduationCap}{Doctoral Thesis}},type=thesis]

\divider

\printbibliography[heading=pubtype,title={\printinfo{\faFileTextO}{Journal Article}},type=article]

\divider

\printbibliography[heading=pubtype,title={\printinfo{\faGroup}{Conference Proceedings}},type=inproceedings]

%% If the NEXT page doesn't start with a \cvsection but you'd
%% still like to add a sidebar, then use this command on THIS
%% page to add it. The optional argument lets you pull up the
%% sidebar a bit so that it looks aligned with the top of the
%% main column.
% \addnextpagesidebar[-1ex]{page3sidebar}

%% Additional content to consider adding:
%% Invited Speaker:
%% - Humanoid Robotics: Ongoing CHallenges and Future Prospects. Swarthmore College, March 2017.
%% - DART 5.0. An overview of the features of DART v5.0 and future directions for ongoing development. Carnegie Mellon University, May 2015.
%% - ROS-Industrial Asia-Pacific Workshop 2018
%% - ROS-Industrial Asia-Pacific Workshop 2019
%% - A*Star 2019
%%
%% Conference reviewer:
%% - Robotics: Science and Systems Conference
%% - IEEE-RAS Int’l Conference on Humanoid Robotics (Humanoids)
%% - IEEE/RSJ Int’l Conference on Intelligent Robots and Systems (IROS)

\end{document}